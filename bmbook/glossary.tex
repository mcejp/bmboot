% Not using the usual Glossaries package, because it doesn't cope well with multiple terms sharing a definition

\section*{Glossary}

\begin{description}
  \item[board support package] \\
  \item[BSP] \hfill \\ A package of software libraries, initialization code, linker scripts etc. provided typically by a hardware vendor to facilitate software development for their platform.

In case of Zynq UltraScale+, the BSP is generated by Vitis based on the FPGA design and additional options selected by the user.

In the past, Bmboot required the BSP during compilation. This is not the case anymore.

  \item[core dump] \\
  \item[core file] \hfill \\ Recorded state of the working memory of a computer program at a specific time, generally when the program has crashed or otherwise terminated abnormally.

  \item[domain] \hfill \\ A CPU core which can potentially become an executor, if not in use by the operating system.

  \item[executor] \\
  \item[executor domain] \hfill \\ A bare-metal core under the control of bmboot.

  \item[exception level] \\
  \item[EL0] \\
  \item[EL1] \\
  \item[EL2] \\
  \item[EL3] \hfill \\ A concept in ARM CPUs which establishes a hierarchy of privilege between different code running on the CPU.
      A higher EL number indicates a higher level of privilege.

  \item[IPC block] \hfill \\ A block of memory dedicated to communication between the :term:`manager` and an :term:`executor`.
      A separate IPC block is allocated to each executor.

  \item[IPI] \\
  \item[Inter-Processor Interrupt] \hfill \\ A mechanism by which one CPU core can trigger an interrupt on another CPU core.

  \item[IRQ] \hfill \\ Interrupt request

  \item[manager] \hfill \\ A process running under Linux, which manages executor domains.

  \item[monitor] \hfill \\ The part of Bmboot which runs on executor CPU cores.

  \item[payload] \hfill \\ The user program which runs on executor CPU cores. It is started by the monitor on user request.

  \item[SMC] \hfill \\ Secure Monitor Call -- a way for the payload to invoke services provided by the monitor
\end{description}
